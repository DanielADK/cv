%%%%%%%%%%%%%%%%%
% This is an example CV created using altacv.cls (v1.1.5, 1 December 2018) written by
% LianTze Lim (liantze@gmail.com), based on the
% Cv created by BusinessInsider at http://www.businessinsider.my/a-sample-resume-for-marissa-mayer-2016-7/?r=US&IR=T
%
%% It may be distributed and/or modified under the
%% conditions of the LaTeX Project Public License, either version 1.3
%% of this license or (at your option) any later version.
%% The latest version of this license is in
%%    http://www.latex-project.org/lppl.txt
%% and version 1.3 or later is part of all distributions of LaTeX
%% version 2003/12/01 or later.
%%%%%%%%%%%%%%%%

%% If you are using \orcid or academicons
%% icons, make sure you have the academicons
%% option here, and compile with XeLaTeX
%% or LuaLaTeX.
% \documentclass[10pt,a4paper,academicons]{altacv}

%% Use the "normalphoto" option if you want a normal photo instead of cropped to a circle
% \documentclass[10pt,a4paper,normalphoto]{altacv}

\documentclass[10pt,a4paper,ragged2e]{altacv}

%% AltaCV uses the fontawesome and academicon fonts
%% and packages.
%% See texdoc.net/pkg/fontawecome and http://texdoc.net/pkg/academicons for full list of symbols. You MUST compile with XeLaTeX or LuaLaTeX if you want to use academicons.

% Change the page layout if you need to
\geometry{left=2cm,right=10cm,marginparwidth=6.8cm,marginparsep=1.2cm,top=1.25cm,bottom=1.25cm}

% Change the font if you want to, depending on whether
% you're using pdflatex or xelatex/lualatex
\ifxetexorluatex
  % If using xelatex or lualatex:
  \setmainfont{Carlito}
\else
  % If using pdflatex:
  \usepackage[utf8]{inputenc}
  \usepackage[T1]{fontenc}
  \usepackage[default]{lato}
\fi

% Change the colours if you want to
\definecolor{VividPurple}{HTML}{000000}
\definecolor{SlateGrey}{HTML}{2E2E2E}
\definecolor{LightGrey}{HTML}{2E2E2E}
\colorlet{heading}{VividPurple}
\colorlet{accent}{VividPurple}
\colorlet{emphasis}{SlateGrey}
\colorlet{body}{LightGrey}

% Change the bullets for itemize and rating marker
% for \cvskill if you want to
\renewcommand{\itemmarker}{{\small\textbullet}}
\renewcommand{\ratingmarker}{\faCircle}

%% sample.bib contains your publications
%% \addbibresource{sample.bib}

\begin{document}
\name{Daniel Adámek}
\tagline{Student \& učitel}
% Profile photo
% \photo{3.3cm}{profile.jpg}
\personalinfo{%
  % Not all of these are required!
  % You can add your own with \printinfo{symbol}{detail}
  \email{daniel.adamek123@icloud.com}
  \phone{+420 733 270 501}
  \github{github.com/DanielADK}
  \location{Praha/Liberec, Česká Republika}
%  \mailaddress{Address, Street, 00000 County}
%  \homepage{marissamayr.tumblr.com/}
%  \twitter{@marissamayer}
%   \orcid{orcid.org/0000-0000-0000-0000} % Obviously making this up too. If you want to use this field (and also other academicons symbols), add "academicons" option to \documentclass{altacv}
}

%% Make the header extend all the way to the right, if you want.
\begin{fullwidth}
\makecvheader
\end{fullwidth}

%% Depending on your tastes, you may want to make fonts of itemize environments slightly smaller
\AtBeginEnvironment{itemize}{\small}

%% Provide the file name containing the sidebar contents as an optional parameter to \cvsection.
%% You can always just use \marginpar{...} if you do
%% not need to align the top of the contents to any
%% \cvsection title in the "main" bar.
\cvsection[experience-cz]{Zkušenosti}

\cvevent{Učitel odborných předmětů}{Střední průmyslová škola elektrotechnická, Praha 2, Ječná 30}{Zář 2021 -- Současnost}{Praha, Česká republika}
\begin{itemize}
\item Výuka webových aplikací a základů informatiky
\item Metodik ICT
\end{itemize}

\divider

\cvevent{Asistent pro výukové materiály}{Fakulta informačních technoligií, ČVUT v Praze}{Úno 2021 -- Čvc 2021}{Praha, Česká republika}
\begin{itemize}
\item Úprava konzistence přednáškových prezentací s výukovými pomůckami předmětu BI-DBS (Databázové systémy),
oprava vzorové databáze a oprava překlepů v jazyce \LaTeX.

\end{itemize}

\divider

\cvevent{Webový administrátor \& vývojář}{Tábořiště Černousy}{Led 2020 -- Současnost}{Černousy, Česká republika}
\begin{itemize}
\item Správa webu a vývoj rezervačního systému tábořiště.
\end{itemize}
 
\divider

\cvevent{Projektant}{AgralPlast s.r.o.}{Čvc 2018 -- Srp 2018}{Liberec, Česká republika}
\begin{itemize}
\item Projekce a pasportizace staveb pro občanskou výstavbu.
\end{itemize}

\divider

\cvevent{Webový administrátor \& vývojář}{T.O.Severka, p.s. Česká Tábornická Unie, z.s.}{Led 2016 -- Současnost}{Liberec, Česká republika}
\begin{itemize}
\item Správa webu, vývoj aplikace pro správu pořádaných akcí a systém elektronických přihlášek účastníků.

\end{itemize}

%\divider

\cvsection{Certifikáty a kurzy}
\smallskip
\begin{itemize}
\smallskip
\item \textbf{CPP: Advanced Essentials in C++} \textit{C++ institute}
\smallskip
\item \textbf{CPA: Programming Essentials in C++} \textit{C++ institute}
\smallskip
\item \textbf{Zadavatel pro žáky s PUP MZ} \textit{Centrum pro zjišťování výsledků vzdělávání}
\smallskip
\item \textbf{Zadavatel} \textit{Centrum pro zjišťování výsledků vzdělávání}
\smallskip
\item \textbf{Database Design} \textit{Oracle Academy}
\smallskip
\item \textbf{Database Programming with SQL} \textit{Oracle Academy}
\smallskip
\item \textbf{Zdravotník zotavovacích akcí} \textit{Český červený kříž}
\end{itemize}

\cvsection{DOVEDNOSTI}

\cvskill{C, C++}{3}
\cvskill{Python}{4}
\cvskill{MySQL, PostgreSQL}{4}
\cvskill{Linux}{4}
\cvskill{AutoCAD}{3}
\cvskill{HTML, CSS, PHP, JS}{4}
%\divider


\clearpage

% \cvsection[page2sidebar]{Publications}

\nocite{*}

% \printbibliography[heading=pubtype,title={\printinfo{\faBook}{Books}},type=book]

% \divider

% \printbibliography[heading=pubtype,title={\printinfo{\faFileTextO}{Journal Articles}}, type=article]

% \divider

% \printbibliography[heading=pubtype,title={\printinfo{\faGroup}{Conference Proceedings}},type=inproceedings]

% %% If the NEXT page doesn't start with a \cvsection but you'd
% %% still like to add a sidebar, then use this command on THIS
% %% page to add it. The optional argument lets you pull up the
% %% sidebar a bit so that it looks aligned with the top of the
% %% main column.
% % \addnextpagesidebar[-1ex]{page3sidebar}


\end{document}
